\documentclass[11pt]{article}
%\usepackage[a4paper, total={7in, 10in}]{geometry}
\usepackage[margin=1in]{geometry}
\usepackage{amsmath, amssymb, braket, dsfont,authblk}
\usepackage{graphicx}
\usepackage{rotating, mathtools}
\usepackage[matrix,arrow]{xy}
\usepackage[numbers]{natbib}
\usepackage[colorlinks]{hyperref}
\hypersetup{linkcolor={blue}}

\usepackage{xifthen}


%%%%% Author comments
\usepackage{color}
\definecolor{ao(english)}{rgb}{0.0, 0.5, 0.0}
\definecolor{americanrose}{rgb}{1.0, 0.01, 0.24}
\definecolor{amber(sae/ece)}{rgb}{1.0, 0.49, 0.0}

\newcommand{\dave}[1]{{\color{ao(english)}\footnotesize{(DA) #1}}}

\newcommand{\remove}[1]{{\color{amber(sae/ece)}\footnotesize{(RM?) #1}}}

\definecolor{amethyst}{rgb}{0.6, 0.4, 0.8}
\newcommand{\ethan}[1]{{\color{amethyst}\footnotesize{(EL) #1}}}


\numberwithin{equation}{section}	


%%%
%%%
%%%
\title{Super lattice models *draft*}
%%%
%%%
\author{Author$^1$,Author$^{2}$}
\affil{
$^1$ Affiliation1}
\affil{
$^2$ Affiliation2}
%%%
%%%

\begin{document}

\maketitle
\begin{abstract}
- fermion condensation/ Ising example

- modular transformations

- super-pivotal category

- exactly solvable Hamiltonian
 
\end{abstract}

\tableofcontents

\section{introduction}
- new phases of matter/other motivation

- fermion condensation as a method to generate super-pivotal categories

- explain how this differs from previous work. 

- emphasize tensoring over endo-morphisms

-  mention string nets from tubes picture/excitations

\section{Ising model}
- introduce Ising TQFT \cite{Lins1994}

- condense fermions

- explain inconsistencies that necessitate spin structure, give physical interpretation of $p+ip$ SC.

- define spin structure

- graphical calculus of condensed theory

- give explicit example of why it's important to tensor over endomorphisms

- Drinfeld center of condensed theory

- oddly isomorphic particles

- Fusion of quasi-particles

- braiding of quasi-particles 

- modular transformations/relation to braiding data, filling in $S^3$ (?)

- state sum???

\section{$SO(3)_6$ and/or sFib?}
Super example with no $q$ type particles in condensed theory.

\section{$\frac{1}{2}$E6}

\dave{could move to appendix or save for another paper}

- introduce 1/2 E6

- explain how to condense fermion despite no braiding

- list idempotents? quasi-particle fusion? braiding? 

\section{super-pivotal fusion category}

- define super pivotal fusion category:

- objects

- fusion rules

- quantum dimensions

- fusion/splitting spaces

- pivotal structure/spherical structure

- theta symbols

- 6j symbols


 
 \section{Hamiltonian}
 
 - Briefly mention string net hamiltonian
 
 - define super string net model 
 
 \section{Conclusion}
 - ....
 \paragraph{Acknowledgements}
 Boulder school, KITP for hosting
 
 

%%%%%%%%%%%%%%%%%%%%%%%%%%%%%%%%%%%%%%%%%%%%%%%%%%%%%%%%%%%%%%%%%%%%%%%%%%%%%%%%
\phantomsection
\addcontentsline{toc}{section}{References}

\bibliography{references}
\bibliographystyle{apsrev4-1}

\clearpage
\appendix
\section{Fusion category review}
In this section we review fusion categories. 
Physically one can think of the diagrams as the world lines of ``particles" in a $1+1$ dimensional theory.
For example this could be considered as the topological theory associated to a gapped boundary of a $2+1$ dimensional topological phase. 

- We label the particles by $\{ a, b, c, \cdots \}$. 
Each particle label represents a super selection sector, i.e., an equivalence class of quantum states that is invariant under local operators \footnote{Formally we should write $X_a$ to denote the equivalence class of states related to $a$ by local operators.}. 
Mathematically this means that $\text{Hom}(a,a) \cong \mathbb{C} id_a$, where $id_a$ is the identity on $a$.
Equivalently we could say that the space of maps from $a$ back to itself is just the complex numbers
\begin{align}
\text{End}(a,a) \cong \mathbb{C}
\end{align}
This is the definition of a {\em simple object} in a fusion category {\cal C}. 
One could consider objects that aren't simple, but we require that in this case that object always decomposes into a direct sum of simple objects, for example, $A = \oplus_i x_i$, where each $x_i$ is simple. 
In this case we have $\text{Hom}(A,A)= \oplus_i \text{Hom}(x_i,x_i)$. 
We will require the fusion category to be {\em semisimple} meaning that each object can always be decomposed into a finite sum of simple objects.

- We can represent this space as a set of labeled points on a a circle or line. (picture).

- Time evolution maps this set of states to another on a circle or a line. During this evolution several processes could ocurr, particles could be created from the vaccuum, they could fuse to the vacuum or they could split. 
Each process is 

- The diagrams we consider will be labeled trivalent graphs. 



\section{appendix 2}


%%%%%%%%%%%%%%%%%%%%%%%%%%%%%%%%%%%%%%%%%%%%%%%%%%%%%%%%%%%%%%%%%%%%%%%%%%%%%%%%

\end{document}